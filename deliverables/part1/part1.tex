\documentclass[a4paper]{article}

\usepackage[english]{babel}
\usepackage[utf8]{inputenc}

\title{TI1705 - Part I}
\author{Gerlof Fokkema - 4257286\\
		Joris Lamers - \textless studentnr\textgreater}
\date{\today}
\setcounter{section}{2}
\newtheorem{thm}{Exercise}


\begin{document}
  \maketitle
  \section{Part I}
    Your introduction goes here!
      
  \subsection{End-to-End Testing}
    \thm Execute the smoke test, with coverage enabled. What overall (line) coverage percentage do you get? Name 2 classes that are not well-tested, and explain why the smoke test does not cover it.

    \thm Study the acceptance scenarios for User Stories 1, 2, and 4. Turn each of them into a test case, as far as possible. To do so, use the approach adopted in LauncherSmokeTest to start the game and trigger specific behavior.
    
    \thm Which functionality of story 2 was hard (or even impossible) to test? Why?
    
    \thm Now try the same for Story 3 (moving monsters). Why are these scenarios hard to test?
  
  \subsection{Boundary Testing}
    \thm Provide a domain matrix for the desired behavior of the boundary values in the withinBorders method.

    \thm Implement the corresponding test by using JUnit’s Parameters annotation.
  
  \subsection{Submit Part I}
    \thm In your report, analyze whether the code is ready for submission: Explain check-style violations that remain (if any), provide a log of all tests passing, and include a brief assessment of the additional adequacy achieved in the jpacman-framework thanks to your new classes. Also reflect on your continuous integration server results, and your commit behavior.
    
    \thm Create a release with appropriate version number in your pom file, git tag and push. Submit the zip and report to CPM as well.

\end{document}
