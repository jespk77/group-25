\documentclass[a4paper]{article}

\usepackage[english]{babel}
\usepackage[utf8]{inputenc}
\usepackage{multirow}

\title{TI1705 - Part II}
\author{Gerlof Fokkema - 4257286}
\date{\today}
\setcounter{section}{3}
\newtheorem{thm}{Exercise}
\setcounter{thm}{8}


\begin{document}
  \maketitle
  \section{Part II}
  
  \subsection{Mocks}
    \begin{thm}
      Write a test suite for the level.MapParser class. Start out with the nice
      weather behavior, in which the board contains expected characters. Use mockito
      to stub the factories, and use mockito to verify that reading a map leads to the
      proper interactions with those factories.
    \end{thm}
    The factories that had to be stubbed were the LevelFactory and the BoardFactory.
    Verifying proper behaviour was done by giving the MapParser a predefined map, for which we know the amount of squares.
    After that we can count how often the MapParser has called specific functions and matching that against the known amounts.

    \begin{thm}
      Extend the test suite to bad weather situations,
      which should raise the proper exceptions.
    \end{thm}
    While extending the test suite to bad weather situations, we encountered a problem with the way EclEmma calculates code coverage.
    EclEmma is not able to calculate code coverage for lines that throw exceptions, therefore the reported code coverage score is significantly lower than the real score.
    
  \subsection{Testing Collisions}
    \begin{thm}
      Analyze requirements (doc/scenarios.md) and derive a decision table for the
      JPacman collisions from it.
    \end{thm}

    \begin{thm}
      Based on the decision table for collisions, derive a JUnit test suite for the
      level.PlayerCollisions class.
    \end{thm}

    \begin{thm}
      Restructure your test suite from exercise 12 so that you can execute the same test
      suite on both PlayerCollision and DefaultPlayerInteractionMap objects.
    \end{thm}
    
    \begin{thm}
      Analyze the increase in coverage compared to the original tests in jpacman-framework,
      and discuss what collision functionality you have covered additionally,
      and which (if any) collision functionality is still unchecked.
    \end{thm}
  
  \subsection{Submit Part II}
    \begin{thm}
      Submit Part II as you submitted as described in Section 3.3.
    \end{thm}

\end{document}
