\documentclass[a4paper]{article}

\usepackage[english]{babel}
\usepackage[utf8]{inputenc}
\usepackage{multirow}
\usepackage{rotating}	

\title{TI1705 - Part II}
\author{Gerlof Fokkema - 4257286}
\date{\today}
\setcounter{section}{3}

\newtheorem{thm}{Exercise}
\setcounter{thm}{8}

\begin{document}
  \maketitle
  \section{Part II}
  
  \subsection{Mocks}
    \begin{thm}
      Write a test suite for the level.MapParser class. Start out with the nice
      weather behavior, in which the board contains expected characters. Use mockito
      to stub the factories, and use mockito to verify that reading a map leads to the
      proper interactions with those factories.
    \end{thm}
    Please refer to \textit{nl.tudelft.jpacman.level.MapParserTest} for the implementation of the JUnit test. \\
    The factories that had to be stubbed were the \textit{LevelFactory} and the \textit{BoardFactory}.
    Verifying proper behaviour was done by giving the MapParser a predefined map, for which we know the amount of squares.
    After that we can count how often the \textit{MapParser} has called specific functions and we can match that against the known amounts.

    \begin{thm}
      Extend the test suite to bad weather situations,
      which should raise the proper exceptions.
    \end{thm}
    While extending the test suite to bad weather situations, we encountered a problem with the way EclEmma calculates code coverage.
    EclEmma is not able to calculate code coverage for lines that throw exceptions, therefore the reported code coverage score is significantly lower than the real score.
  
  \newpage
  \subsection{Testing Collisions}
    \begin{thm}
      Analyze requirements (doc/scenarios.md) and derive a decision table for the
      JPacman collisions from it.
    \end{thm}
    After analyzing the requirements in \textit{scenarios.md} and the implementation in \textit{PlayerCollisions}
    we concluded that Units are subdivided in 2 types for collision maps: the \textit{mover} and the \textit{movedInto} unit. \\
    In JPacman, there's 2 types of Units that can be a \textit{mover}:
    \begin{itemize}
      \item a \textit{Player}
      \item a \textit{Ghost}
    \end{itemize}
    A \textit{mover} can then collide with Units of all types:
    \begin{itemize}
      \item a \textit{Player}
      \item a \textit{Ghost}
      \item a \textit{Pellet}
    \end{itemize}
    Putting all this together we get the following decision table:
    \begin{table}[h]
      \begin{tabular}{|l|l|l|c|c|c|c|c|c|}
        \cline{4-9}
        \multicolumn{3}{}{}	& \multicolumn{6}{|c|}{Rules}		\\ \hline
        \multirow{5}{*}{\rotatebox{90}{Conditions}}
          & \multirow{2}{*}{Mover}	& Player	& Y	& Y	& Y	& -	& -	& -	\\ \cline{4-9}
          &				& Ghost		& -	& -	& -	& Y	& Y	& Y	\\ \cline{2-9}
          & \multirow{3}{*}{CollidedOn}	& Player	& Y	& -	& -	& Y	& -	& -	\\ \cline{4-9}
          &				& Ghost		& -	& Y	& -	& -	& Y	& -	\\ \cline{4-9}
          &				& Pellet	& -	& -	& Y	& -	& -	& Y	\\ \hline \hline
	\multirow{3}{*}{\rotatebox{90}{Actions}}
          & \multicolumn{2}{|l|}{Player dies}		&	& X	&	& X	&	&	\\ \cline{2-9}
          & \multicolumn{2}{|l|}{Pellet is taken}	&	&	& X	&	&	&	\\ \cline{2-9}          
          & \multicolumn{2}{|l|}{Player gets points}	&	&	& X	&	&	&	\\ \hline
      \end{tabular}
    \end{table}

    \begin{thm}
      Based on the decision table for collisions, derive a JUnit test suite for the
      level.PlayerCollisions class.
    \end{thm}
    Please refer to \textit{nl.tudelft.jpacman.level.PlayerCollisionsTest} for the implementation of the JUnit test.

    \begin{thm}
      Restructure your test suite from exercise 12 so that you can execute the same test
      suite on both PlayerCollision and DefaultPlayerInteractionMap objects.
    \end{thm}
    To be able to execute the same test on multiple CollisionMaps, we've chosen to implement PlayerCollisionsTest as a parameterized JUnit test.
    This means we won't be able to use the MockitoJUnitRunner. Therefore we initialize Mockito manually instead.
    
    \begin{thm}
      Analyze the increase in coverage compared to the original tests in jpacman-framework,
      and discuss what collision functionality you have covered additionally,
      and which (if any) collision functionality is still unchecked.
    \end{thm}
    
    The coverage percentage we reported in the previous lab exercise (excluding test cases) was 79\%.
    After implementing the test cases for this lab exercise, the overall coverage percentage has risen to 86\%.
    An overview of what has changed:
    \begin{table}[h]
      \begin{tabular}{|l|c|c|}
        \cline{2-3}
        \multicolumn{1}{}{}		& Original coverage \%	& New coverage \%	\\ \hline
        PlayerCollisions		&			&			\\ \hline
        CollisionInteractionMap		&			&			\\ \hline
        DefaultPlayerInteractionMap	&			&			\\ \hline
        Overall				&			&			\\ \hline
      \end{tabular}
    \end{table}
    
        In the original tests in jpacman-framework, the coverage for PlayerCollisions was 91\% and
    the coverage for both CollisionInteractionMap and DefaultPlayerInteractionMap was 0\%.
    The coverage percentage for CollisionInteractionMap has risen to 97\% and for 
  
  \subsection{Submit Part II}
    \begin{thm}
      Submit Part II as you submitted as described in Section 3.3.
    \end{thm}

\end{document}
